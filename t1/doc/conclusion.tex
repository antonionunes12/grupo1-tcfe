\section{Final Conclusion and General Notes}
\label{sec:conclusion}

To conclude, we can realise that the results presented in Tables \ref{tab:nodal}, \ref{tab:mesh} and \ref{tab:op} are similar, with exception of the last represented decimal places (due to rounding), which is satisfying. Despite that we must be more insightful and analyst towards the data given along this report. \par
So let us firstly discuss about the linearity of the circuit. Because all the components are linear, it is always expected a steady-state solution, invariable with respect to time. Also, another consequence of this is that we should expect no "butterfly-effects", i.e. a little nuance in the given data should not scale up to enormous disparities in the outputs. \par
In second it should be pointed out that although the circuit analysis is simple, the equations lead to complicated algebra, which without Octave would be very time consuming and easy to miscalculate. The best evidence of that is the \emph{symbolic} solution matrix from both matrix equations (\ref{eq:node}), (\ref{eq_mesh}), which is not presented in this report due to its size and each entry complex expression, as mentioned in Section \ref{subsec:results}. \par
Finally, as already discussed in previous Sections, because the simulation software uses MNA, an equivalent method as the ones used to analyse theoretically, the results of both approaches (theoretical and practical) must be the same. As an exercise, the circuit could also be analysed using the Superposition Theorem, its \textit{Thévenin}'s or \textit{Norton}'s equivalent and of course, we must expect no difference in the results. 
