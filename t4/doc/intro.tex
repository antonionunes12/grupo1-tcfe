\section{Introduction}
\label{sec:introduction}

In this laboratory, we analysed in a theoretical approach as well as using software simulation, an AC/DC converter, composed with an Envelope Detector Circuit and a Voltage Regulator Circuit. In this report, a software simulation and theoretical analysis will be stacked up against each other. This assignement allowed us to deal with important concepts such as \textbf{diodes} and its diverse utility in circuits. In Figure \ref{fig:circuitol3} the stated circuit is presented. 

A theoretical analysis of the circuit will be presented combining DC and incremental analysis, as well as a RC natural solution (for the envelope detector and voltage ripple calculation), giving us some insights on the non-linearity behaviour of the circuit. Also, it is important to notice that for better teory-experimental marriage, an ideal diode model is combined with a voltage source $V_{ON}$ to simulate the potencial drop across a diode.  
Regarding the DC and incremental analysis of the voltage regulator circuit, we superimposed both solutions referred and studied the precision (time average of $V_{out}$ - which should be as close to 12$V$ as possible) and the voltage ripple of the output which stands for the difference between maximum and minimum peaks of the output voltage.

Simultaneously, the circuit is analysed by computational simulation tools, via \textit{Ngspice}, and the results are compared to the theoretical results obtained, in Section \ref{sec:analysis}. The conclusions of this study are outlined in Section \ref{sec:conclusion}.

\begin{figure}[h] \centering
\includegraphics[width=1\linewidth]{CircuitL3.pdf}
\caption{AC/DC Converter used.}
\label{fig:circuitol3}
\end{figure}

\clearpage
