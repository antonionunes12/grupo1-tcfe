\section{Final Conclusion and General Notes}
\label{sec:conclusion}

As a conclusion, we can state that, unlike previous lab assignments, there is not a major degree of similarity between both analysis. This was expected due to the fact that the circuit is non-linear because of the presence of diodes. The model used by \textit{Ngspice} is far more complex than the theoretical model used. Despite these differences, the theoretical analysis can be quite accurate. Both \textit{Ngspice} and \textit{Octave} plots are similar. The only substancial error is in the voltage ripple and deviation calculations, however, because this AC/DC converter characteristics are so prone to subtle deviations, this is expected. 

Although we tried to use exactly the the same values for the diode parameters ($V_{on}, \eta, Is, V_{T}$) that Ngspice uses for its diode model, the ideal linear model implemented in the theoretical analysis still was no match to the non-linear one implemented by \textit{Ngspice}, which gave off results that can be assumed to be more in par with reality.

With all this in mind, this laboratory enabled us to deepen our knowledge regarding the diode, its different models and how it can be implemented to develop circuits with various purposes.
