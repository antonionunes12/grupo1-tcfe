\section{Simulation Analysis}
\label{sec:simulation}

\subsection{Operating Point Analysis}

In this section, the results given by the \textit{Ngspice} simulation are presented in Figure \ref{fig:alternative_circuit_for_ngspice}. Table \ref{tab:op} shows the simulated operating point results for the circuit
under analysis.

\begin{table}[h]
  \begin{minipage}{0.4\textwidth}
  \centering
  \begin{tabular}{|l|c|}
    \hline    
    {\bf Name} & {\bf Value [$A$ or $V$]} \\ \hline
    @gb[i] & -2.54636e-04\\ \hline
@id[current] & 1.012814e-03\\ \hline
@r1[i] & 2.427540e-04\\ \hline
@r2[i] & -2.54636e-04\\ \hline
@r3[i] & -1.18817e-05\\ \hline
@r4[i] & 1.233308e-03\\ \hline
@r5[i] & -1.26745e-03\\ \hline
@r6[i] & 9.905539e-04\\ \hline
@r7[i] & 9.905539e-04\\ \hline
v(1) & 5.170385e+00\\ \hline
v(2) & 4.919669e+00\\ \hline
v(3) & 4.390981e+00\\ \hline
v(4) & 4.955840e+00\\ \hline
v(5) & 8.823660e+00\\ \hline
v(6) & -2.03701e+00\\ \hline
v(7) & -2.03701e+00\\ \hline
v(8) & -3.04368e+00\\ \hline

  \end{tabular}
  \caption[Simulation of Currents and Voltages]{Simulation of Currents and Voltages\footnotemark}
  \label{tab:op}
  \end{minipage}
  \hfill
  \begin{minipage}{0.53\textwidth}
    \centering
    \includegraphics[scale=0.42]{dummy_voltage_source.pdf}
    \captionof{figure}{Circuit with Dummy Voltage Source}
    \label{fig:alternative_circuit_for_ngspice}
  \end{minipage}
\end{table}

\footnotetext{Variables preceded by @ denote {\em currents} and are expressed in Ampere; the remaining variables are {\it voltages} and are expressed in Volt. Gb represents the voltage controlled current source.}
It should be stated that one node was added between resistors R6 and R7 in order to add a dummy 0V-voltage source in that location. This was done as a work-around the mechanics of \emph{Ngspice} \footnote{Ngspice software only accepts controlling currents that pass in a voltage source. In our example the controling current $I_c$ passes through a resistor.} to enable the voltage source $V_c$ to be controlled by the current passing through $R_6$, $I_c$, and has no real impact on the circuit itself since that, by definition an ideal voltage source has a $0\Omega$ resistance, and so it doesn't change the current neither the voltage drop between the nodes. This is the reason why, on Table \ref{tab:op}, we have v(6) and v(7) outputting the same value.

\clearpage



