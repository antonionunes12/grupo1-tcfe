\section{Introduction}
\label{sec:introduction}

In this laboratory, we analysed in a theoretical approach as well as using software simulation, a first-order forced circuit, in particular, RC with Forced Sinusoidal Voltage. With this method, software simulation and theoretical analysis can be stacked up against each other more easily. It allowed us to deal with important concepts such as the \textbf{impedance} (and its inverse, admittance), \textbf{phasors} (with amplitudes and phases), as well as \textbf{stationary, transient and frequency response} analysis. In Figure \ref{fig:circuitol2} the stated circuit is presented. 

A theoretical analysis of the circuit will be presented using Nodal Analysis giving us some insights on the stationary behaviour of the circuit (\ref{subsec:stat}), before the time starts counting ($t<0$). Also, Nodal Analysis is used in its phasor form to derive the forced solution for node voltages.  
Later, we superimposed both solutions referred and studied the response of the circuit regarding the change in frequency (either its Magnitude, as well as, Phase).


At the same time, the circuit is analysed by computational simulation tools, via \textit{Ngspice}, and the results are compared to the theoretical results obtained, in Section \ref{sec:analysis}. The conclusions of this study are outlined in Section \ref{sec:conclusion}.

\begin{figure}[h] \centering
\includegraphics[width=0.6\linewidth]{circuitol2.pdf}
\caption{RC Circuit.}
\label{fig:circuitol2}
\end{figure}

\clearpage
