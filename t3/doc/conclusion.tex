\section{Final Conclusion and General Notes}
\label{sec:conclusion}

As a conclusion, we can state that there is a major degree of similarity between both analysis. This was expected due to the fact that the circuit is linear and that the overall component complexity is reduced (in fact, this circuit belongs to one of the simplest circuit categories one can have - a 1st order forced circuit). The model used by \textit{Ngspice} is almost entirely equal to the one we used on the theoretical analysis. This was the main reason as to why we were able to obtain near-zero errors, related only to the approximations both tools had to make.

Although difficult to conceptualise, the understanding of the effects studied in this laboratory was greatly aided by the use of the simulation tool. Seeing what was expected to happen was able to give some insights as to how some of the variables intertwined. The graphs were also a great tool to visualize what was happening. Some concepts as the \textbf{filter} were better understood by examining the frequency response of the capacitor. Another important notion was the understanding of \textbf{charge/discharge} on a capacitor.

In conclusion, it was an extremely important step, if not a necessary one, to deepen the knowledge on these types of circuits, in particular, a RC one.
