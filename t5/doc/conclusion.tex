\section{Final Conclusion and General Notes}
\label{sec:conclusion}

As a conclusion, we can state that, alike the previous lab assignment, there is not a major degree of similarity between both analysis, in terms of precision. This was expected due to the fact that the circuit is non-linear and the model used by \textit{Ngspice} is far more complex than the theoretical model used - the Incremental Analysis one, presented on classes, only includes 2 resistors and a dependent current source. Despite these differences, as previously discussed, the theoretical model gives an overview of the behaviour, so it is useful when we don't have any simulation tools available or even to quickly verify the simulation results obtained.

With all this in mind, this laboratory enabled us to deepen our knowledge regarding BJTs and how they can be implemented to develop circuits with various purposes - in our case, an AUDIO Amplifier, even thought the real model amplifiers are far more complex than the circuit implemented, achieving gains of around 115 dB. Besides, we used new concepts such as the \textit{Time constant method}, the incremental models, the input and output impedances and the gain (these, eventhough already familiar, allowed us to gain flexibility and easiness).

Regarding our results, especially the simulation ones, the main goal was to have a high gain and a large enough bandwidth that would cover at least 20Hz to 20kHz, since this is the human hearing range. We can state that we obtained results that more than cover said range and would be suitable for a real audio amplifier. Given that, one possible improvement to our circuit would be to increase the gain even more, which we were not able to do.
