\section{Introduction}
\label{sec:introduction}

% state the learning objective 
\indent The aim of this laboratory assignment is to study the behaviour of a circuit containing exclusively linear components. In Figure \ref{fig:circuitol1} the stated circuit is presented. Despite being a simple circuit, it is a perfect example since our ultimate goal is to experiment some important Circuit Theory analysing tools, presented in Section \ref{sec:analysis}, so that software simulation and theoretical analysis can be stacked up against each other more easily.

In Section~\ref{sec:analysis}, a theoretical analysis of the circuit is
presented, both by the Node Analysis (\ref{subsec:node_analysis}) and Mesh Analysis (\ref{subsec:mesh_analysis}) methods, giving us some insights on different, although equivalent forms of exploring a linear circuit. In Section \ref{sec:simulation}, the circuit is analysed by computational simulation tools, via \textit{Ngspice}, and the results are compared to the theoretical results obtained in Section \ref{sec:analysis}. The conclusions of this study are outlined in Section \ref{sec:conclusion}.

\begin{figure}[h] \centering
\includegraphics[width=0.6\linewidth]{circuitol1.pdf}
\caption{Linear Circuit with Resistors, Dependent and Independent Sources.}
\label{fig:circuitol1}
\end{figure}

\clearpage
