\section{Simulation Analysis}
\label{sec:simulation}

\subsection{Operating Point Analysis}

In this section, we present the results given by the \emph{Ngspice} simulation of the circuit in Figure \ref{fig:alternative_circuit_for_ngspice}.Table~\ref{tab:op} shows the simulated operating point results for the circuit
under analysis.

\begin{table}[h]
  \centering
  \begin{tabular}{|l|r|}
    \hline    
    {\bf Name} & {\bf Value [A or V]} \\ \hline
    \input{op_tab}
  \end{tabular}
  \caption[Simulation of Currents and Voltages]{Simulation of Currents and Voltages\footnotemark}
  \label{tab:op}
\end{table}
\footnotetext{Variables preceded by @ denote {\em currents} and are expressed in Ampere; the remaining variables are {\it voltages} and are expressed in Volt. Gb represents the current controlled current source.}
It should be stated that one node was added between resistors R6 and R7 in order to add a dummy 0V-voltage source in that location. This was done as a work-around the mechanics of \emph{Ngspice} to enable the voltage source $V_c$ to be controlled by the current passing through $R_6$, $I_c$, and has no real impact on the circuit itself since that, by definition a voltage source has $0\Omega$ resistance, and so it doesn't change the current neither the voltage drop between the nodes. This is the reason why, on Table \ref{tab:op}, we have v(6) and v(7) outputting the same value.

\begin{figure}[h]
    \centering
    \includegraphics[scale=0.32]{dummy_voltage_source.pdf}
    \caption{Circuit with Dummy Voltage Source}
    \label{fig:alternative_circuit_for_ngspice}
\end{figure}

\clearpage



