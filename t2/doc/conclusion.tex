\section{Final Conclusion and General Notes}
\label{sec:conclusion}

To conclude, we can realise that the results presented in Tables \ref{tab:nodal} and \ref{tab:mesh} are equal to the ones displayed in Table \ref{tab:op} and that's the reason why there's no error analysis in the present report.  Despite that, we must be more insightful and analyst towards the data given along this report. \par
So let us firstly discuss about the linearity of the circuit. Because all the components are linear, a steady-state solution is always expected, invariable with respect to time. Also, another consequence of this, is that we should not attend any "butterfly-effects", \textit{i.e.} a little nuance in the given data should not scale up to enormous disparities in the outputs, which is one factor as to why the results fit so well with the simulation. \par
Secondly, it should be pointed out that although the circuit analysis is simple, the equations lead to complicated algebraic systems, which without \textit{Octave} would be very time consuming and easy to miscalculate. The best evidence of that is the \emph{symbolic} solution matrix from both matrix equations (\ref{eq:node}) and (\ref{eq_mesh}), which is not presented in this report due to its size and the complexity of each entry, as mentioned in Section \ref{subsec:results}. \par
Finally, as already discussed in previous Sections, because the simulation software uses MNA, an equivalent method as the ones used to analyse theoretically, the results of both approaches (theoretical and practical) must be the same. As an exercise, the circuit could also be analysed using the Superposition Theorem, its \textit{Thévenin}'s or \textit{Norton}'s equivalents and of course, we must expect no difference in the results. These analysis are beyond the extent of this report. 

Overall, this project was a beneficial tool to develop our theoretical circuit analysis and software programming capabilities. 
